\documentclass[
	% -- opções da classe memoir --
%	article,			% indica que é um artigo acadêmico
	11pt,				% tamanho da fonte
	openright,
	twoside,			% Oposto a oneside
	a4paper,			% tamanho do papel. 
	english,			% idioma adicional para hifenização
	french,
	brazil,				% o último idioma é o principal do documento
	sumario=tradicional
	]{abntex2}

\usepackage{lmodern}			% Usa a fonte Latin Modern
\usepackage[T1]{fontenc}		% Selecao de codigos de fonte.
\usepackage[utf8]{inputenc}		% Codificacao do documento (conversão automática dos acentos)
\usepackage{indentfirst}		% Indenta o primeiro parágrafo de cada seção.
\usepackage{nomencl} 			% Lista de simbolos
\usepackage{color}				% Controle das cores
\usepackage{graphicx}			% Inclusão de gráficos
\usepackage{microtype} 			% para melhorias de justificação
% Tabelas longas -> + de uma página
\usepackage{longtable}

\usepackage[brazilian,hyperpageref]{backref}	% Paginas com as citações na bibl
\usepackage[alf]{abntex2cite}					% Citações padrão ABNT

\renewcommand{\backrefpagesname}{Citado na(s) página(s):~}
\renewcommand{\backref}{}
\renewcommand*{\backrefalt}[4]{
	\ifcase #1 %
		Nenhuma citação no texto.%
	\or
		Citado na página #2.%
	\else
		Citado #1 vezes nas páginas #2.%
	\fi}%
% ---

\titulo{Apostila de conceitos básicos de Orientação Objeto}
\autor{Paulino Ng\thanks{PANG CO.\\paulino.ng@gmail.com}
}
\local{Brasil}
\data{2019, v-0.5.1}

% informações do PDF
\makeatletter
%\hypersetup{
%     	%pagebackref=true,
%		pdftitle={\@title}, 
%		pdfauthor={\@author},
%    	pdfsubject={Apostila de conceitos básicos de OO},
%	    pdfcreator={PDFLaTeX with abnTeX2 on texmaker},
%		pdfkeywords={OO}{Scott Ambler}{UML}{Java EE}, 
%		colorlinks=true,       		% false: boxed links; true: colored links
%    	linkcolor=blue,         	% color of internal links
%    	citecolor=blue,        		% color of links to bibliography
%    	filecolor=magenta,      	% color of file links
%		urlcolor=blue,
%		bookmarksdepth=4
%}
\makeatother
% --- 

\makeindex

% Altera as margens padrões
% ---
\setlrmarginsandblock{2cm}{2cm}{*}
\setulmarginsandblock{2cm}{2cm}{*}
\checkandfixthelayout

% O tamanho do parágrafo é dado por:
\setlength{\parindent}{1.0cm}

% Controle do espaçamento entre um parágrafo e outro:
\setlength{\parskip}{0.2cm}  % tente também \onelineskip

% Espaçamento simples
\SingleSpacing

\begin{document}
\selectlanguage{brazil}

% Retira espaço extra obsoleto entre as frases.
\frenchspacing 

%\maketitle

\imprimircapa

\pdfbookmark[0]{\contentsname}{lof}
\listoffigures*
%\cleardoublepage*
\newpage 

\pdfbookmark[0]{\contentsname}{lot}
\listoftables*
%\cleardoublepage*
\newpage

\pdfbookmark[0]{\contentsname}{toc}
\tableofcontents*
%\cleardoublepage*
\newpage 

% resumo em português
\begin{resumoumacoluna}
Este texto objetiva introduzir alguns dos principais conceitos da tecnologia Orientada a Objetos, OO.
O texto é fortemente baseado no capítulo 2 do livro \cite{Ambler:TOP:3ed}, com acréscimos baseados em \cite{uml:j2ee} e outros textos.
%
\vspace{\onelineskip}
\noindent
\textbf{Palavras-chave}: OO. UML. JEE.
\end{resumoumacoluna}

\textual

% try to generate a table of contents after summary
%\tableofcontents

\chapter{Conceitos Básicos de Orientação a Objetos}
\section{Introdução}
\addcontentsline{toc}{section}{Introdução}

\input{cboo.tex}

\chapter{Codificação em Java}

Veremos neste capítulo alguns exemplos de código Java para os conceitos vistos até o presente e mais alguns novos bastante usados.

\section{Exemplo de MVC - \emph{Model-View-Controller}}

\begin{figure}[h]
\begin{center}
\includegraphics[scale=0.7]{MVC.png} 
\caption{Arquitetura \emph{Model-View-Controller}} \label{fig:mvc}
\end{center}
\legend{Fonte: \citeonline{uml:j2ee}}
\end{figure}

O exemplo a seguir vem de \citeonline{just:java2}. Nele mostramos um exemplo simples da arquitetura MVC usada para aplicações com GUI, interface usuário gráfica. A figura \ref{fig:mvc} mostra os princípios desta arquitetura. O \emph{modelo} na arquitetura encapsula a lógica de negócio da solução do problema resolvido pela aplicação. A \emph{visão} contém o código que exibe o estado do modelo para o usuário e a interface para o usuário interagir com a aplicação. O \emph{controlador} recebe os eventos de entrada do usuário que vêm pela interface com o usuário, a \emph{visão}. Estes eventos podem modificar o que está sendo exibido ou como está sendo exibido pela \emph{visão}. Estes eventos do usuário também podem provocar mudanças no estado do \emph{modelo}. Mudanças no estado do \emph{modelo} podem gerar notificações de alterações para a \emph{visão}.

O programa é simplesmente um relógio em Java. A figura \ref{fig:clock} mostra o diagrama de classes da aplicação. Compare com o diagrama de dependência da figura \ref{fig:clsDep}.

\begin{figure}[h]
\begin{center}
\includegraphics[scale=0.6]{clock.png} 
\caption{Diagrama de classes do relógio digital.} \label{fig:clock}
\end{center}
\end{figure}

\begin{verbatim}
public class TimeStamp {
    private int hrs;
    private int mins;
    private int secs;
    
    public void fillTimes() {
        java.util.Calendar now;
        now = java.util.Calendar.getInstance();
        hrs = now.get(java.util.Calendar.HOUR_OF_DAY);
        mins = now.get(java.util.Calendar.MINUTE);
        secs = now.get(java.util.Calendar.SECOND);
    }
    public getHrs() { return hrs; }
    public getMins() { return mins; }
    public getSecs() { return secs; }
}
\end{verbatim}

A classe \emph{TimeStamp} tem como atributos a hora, os minutos e os segundos e um método que atualiza os valores dos atributos com a ajuda da classe \emph{java.util.Calendar} do sistema. A \emph{TimeStamp} modela a ``lógica de negócio'' de um relógio. A classe \emph{Calendar} do pacote \emph{java.util} é uma classe que vem na biblioteca padrão distribuída com o JDK, kit de desenvolvimento do Java. Observe que os atributos estão ocultos dos usuários da classe \emph{TimeStamp}, os valores deles podem ser obtidos através dos métodos \textit{getters} dos atributos. Se mais tarde, for decidido que se deseja mais alterar estes atributos, pode se reprogramar os métodos \textit{getters} de acordo com as mudanças e os usuários da classe não precisam saber que os atributos foram modificados. Dizemos que isolamos os usuários da classe \emph{TimeStamp} das modificações nos atributos ocultos. Precisamos, entretanto, manter o contrato de que os métodos \emph{getHrs()}, \emph{getMins()} e \emph{getSecs()} continuam a fornecer as horas, os minutos e os segundos.

\begin{verbatim}
public class ClockView extends javax.swing.JFrame {
    private javax.swing.JLabel tLabel = new javax.swing.JLabel();
    
    public ClockView() {
        setDefaultCloseOperation(javax.swing.WindowConstants.EXIT_ON_CLOSE);
        setSize(95,45);
        getContentPane().add(tLabel);
        refreshTimeDisplay();
    }
    
    protected String getDigitAsString(int i) {
        String str = Integer.toString(i);
        if (i<10) str = "0" + str;
        return str;
    }
    
    public void refreshTimeDisplay() {
        TimeStamp ts = new TimeStamp();
        ts.fillTimes();
        String display = getDigitAsString(ts.getHrs()) + ":"
                     + getDigitAsString(ts.getMins()) + ":"
                     + getDigitAsString(ts.getSecs());
        tLabel.setText("  " + display);
        tLabel.repaint();
    }
}
\end{verbatim}

% os códigos podem ir para minipáginas flutuantes/ou presas
A classe \emph{ClockView} é a nossa classe de visualização. Ela herda da classe \emph{javax.swing.JFrame}. O pacote \emph{javax.swing} é parte da nova biblioteca do Java para a criação de interfaces gráficas para usuários para aplicações de \textit{desktop}. Ao herdar de \emph{JFrame}, um objeto \emph{ClockView} serve de base para a janela onde estará o relógio. O único elemento gráfico que é adicionado ao painel da janela é um \emph{JLabel}. Um objeto \emph{JLabel} permite a exibição de um texto. Para transformar a informação do objeto TimeStamp em texto a ser exibido (variável \emph{display}), a classe ClockView tem o método auxiliar \emph{getDigitAsString(int)} que converte um inteiro em texto. Observe que não foram tomados cuidados para garantir que o inteiro só tem 2 dígitos. Como sabemos que os dados vem da TimeStamp corretamente, não é necessário fazer este tipo de checagem.

\begin{verbatim}
public class Clock {
    public static void main(String[] argv) {
        new Clock().go();
    }
    public void go() {
        ClockView cv = new ClockView();
        cv.setVisible(true);
        try {
            for (;;) {
                cv.refreshTimeDisplay();
                Thread.sleep(500);
            }
        } catch (Exception e) {
            System.out.println("Erro: " + e);
        }
    }
}
\end{verbatim}

A classe \emph{Clock} é a controladora da aplicação. Como o relógio digital é muito simples, tudo que ela tem é o método \emph{main()} que repete para sempre (\emph{loop} infinito) a atualização da tela do relógio e espera por meio segundo (dormindo). A espera é importante para liberar o processador do computador para outras aplicações. A razão de esperar por meio segundo e não por um segundo é que o erro de exibição no caso da espera de um segundo pode ser de até um segundo. No pior caso o programa leu as horas com o objeto TimeStamp justo quando ia mudar o segundo e mostra o valor anterior, só após o tempo de espera de 1s é que a tela vai ser atualizada. Com meio segundo de espera, o erro máximo é de 0,5s.

%\section{Programação J2EE}


% ---
% Finaliza a parte no bookmark do PDF, para que se inicie o bookmark na raiz
% ---
\bookmarksetup{startatroot}% 
% ---

% ---
% Conclusão
% ---
%\section*{Considerações finais}
%\addcontentsline{toc}{section}{Considerações finais}
%
%Texto qualquer.
%
%\begin{citacao}
%Mais texto genérico.
%\end{citacao}
%
%Texto.

% ----------------------------------------------------------
% ELEMENTOS PÓS-TEXTUAIS
% ----------------------------------------------------------
\postextual

% ---
% Título e resumo em língua estrangeira
% ---

% \twocolumn[    		% INICIO DE ARTIGO EM DUAS COLUNAS

% titulo em inglês
%\titulo{Canonical academic article model with \abnTeX}
%\emptythanks
%\maketitle
%
%% resumo em português
%\renewcommand{\resumoname}{Abstract}
%\begin{resumoumacoluna}
% \begin{otherlanguage*}{english}
%   According to ABNT NBR 6022:2003, an abstract in foreign language is a back
%   matter mandatory element.
%
%   \vspace{\onelineskip}
% 
%   \noindent
%   \textbf{Keywords}: latex. abntex.
% \end{otherlanguage*}  
%\end{resumoumacoluna}

% ]  				% FIM DE ARTIGO EM DUAS COLUNAS
% ---

% ----------------------------------------------------------
% Referências bibliográficas
% ----------------------------------------------------------
\bibliography{oo_refs}
\

\end{document}
